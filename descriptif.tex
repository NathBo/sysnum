\documentclass[12pt,a4paper,french]{article}
\usepackage[utf8]{inputenc}
\usepackage[french]{babel}
\selectlanguage{french}
\usepackage[T1]{fontenc}
\usepackage{amssymb}
\usepackage{amsmath}
\usepackage{tabularx}

\title{Projet de simulateur de microprocesseur}
\author{Nathan Boyer, Matteo Torrents, Antoine Cuvelier, Grégoire Le Corre}


\begin{document}

\maketitle

\section*{Introduction}

On cherche à coder un simulateur de microprocesseur en Ocaml.

Pour cela, on va utiliser un simulateur de netlist qui éxécutera des netlists écrites en miniJazz.

Cette netlist peut éxécuter les opérations arithmétiques de base (and, or etc \ldots) sur un bit, manipuler des bus de bits (splice, concat \ldots)

De plus, l'instruction reg permet de se prendre la valeur du cycle précédent d'une variable. L'instruction RAM permet d'écrire et de lire dans la mémoire
(par soucis de réalisme, on considérera qu'il n'existe qu'une seule RAM). L'instruction ROM, quant à elle, permet de lire le contenu d'un fichier composé de 0 et de 1

On va alors programmer un langage assembleur qui nous permettra de communiquer avec notre microprocesseur. Notamment, on programmera une horloge graphique (càd qu'on s'occupera
de l'affichage des pixels depuis notre microprocesseur) qui compte en année, mois, jour, heure, minute et seconde jusqu'à l'année 9999.
pour tester notre microprocesseur.


\section{Les registres}

Tous les calculs du microprocceseur seront faits sur des registres de 16 bits.

Il y a en tout 16 registres.


\section{Représentation des instructions}

Le microprocesseur lit les instructions qu'il doit exécuter depuis un fichier (lu avec l'instruction ROM).

Chaque instruction fait 32 bits, peut prendre en argument jusqu'à 2 numéros de registre et une constante. On doit donc stocker une instruction sur deux registres, le deuxième correspondant toujours à une constante (non initialisée si l'instruction n'utilise pas de constante). Une instruction est donc découpée en 

\begin{itemize}
    \item 4 bits du code de l'instruction
    \item deux blocs de quatre bits renseignant les registres en argument
    \item un deuxième registre de constante
\end{itemize}


\section{Du langage assembleur à la ROM}

Un parseur menhir lit les instructions et écrit dans un fichier lisible par la ROM les différentes instructions.

Voici une liste des différentes instructions, de leur code en binaire, et de leur effet.

\;



\begin{tabularx}{15cm}{|c|p{4cm}|X|}
    \hline
    nom & code & effet \\
    \hline
    stop & 0000 & stoppe le programme \\
    \hline
    add r1 r2 & 0010 & r1 <- r1 + r2 \\
    \hline
    sub r1 r2 & 0110 & r1 <- r1 - r2 \\
    \hline
    movr r1 r2 & 1010 & r1 <- r2 \\
    \hline
    movc r1 c & 0011 & r1 <- c \\
    \hline
    jump r1 r2 c & 0001 & on saute à la ligne c si r1 = r2 \\
    \hline
    getram r1 r2 c & 0101 & r1 <- RAM(r2 + c) \\
    \hline
    setram r1 r2 c & 1001 & RAM(r2 + c) <- r1 \\
    \hline
    rom r1 r2 c & 1101 & r1 <- ROM(r2 + c) \\
    \hline
\end{tabularx}


\;


Le bit de poids faible indique si l'instruction contient une constante, le deuxième bit de poids faible si l'instruction appelle un ou deux registres.
On numérote ensuite les instructions de chaque catégorie en les énumérant. 

\section{Affichage graphique}

L'affichage de l'horloge se fait dans une fenêtre graphique de dimension. La date est affichée sous la forme 

\textit {heures}: \textit {minutes} : \textit {secondes} \textit {jour} / \textit {mois}/ \textit {année} 

où les passages en italiques représentent les valeurs machine.
Concrètement on alloue une partie de la mémoire pour l'affichage; chaque bit en mémoire représente un mégapixel (noir ou blanc). Le microprocesseur modifie cette espace mémoire à chaque cycle et Ocaml affiche la fenêtre bit à bit.




\end{document}
