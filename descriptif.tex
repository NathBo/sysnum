\documentclass[12pt,a4paper,french]{article}
\usepackage[utf8]{inputenc}
\usepackage[french]{babel}
\selectlanguage{french}
\usepackage[T1]{fontenc}
\usepackage{amssymb}
\usepackage{amsmath}
\usepackage{tabularx}

\title{Projet de simulateur de microprocesseur}
\author{Nathan Boyer, Matteo Torrents, Antoine Cuvelier, Grégoire Lecorre}


\begin{document}

\maketitle

\section*{Introduction}

On cherche à coder un simulateur de microprocesseur en Ocaml.

Pour cela, on va utiliser un simulateur de netlist qui éxécutera des netlists écrites en miniJazz.

Cette netlist peut éxécuter les opérations arithmétiques de base (and, or etc \ldots) sur un bit, manipuler des bus de bits (splice, concat \ldots)

De plus, l'instruction reg permet de se prendre la valeur du cycle précédent d'une variable. L'instruction RAM permet d'écrire et de lire dans la mémoire
(par soucis de réalisme, on considérera qu'il n'existe qu'une seule RAM). L'instruction ROM, quant à elle, permet de lire le contenu d'un fichier composé de 0 et de 1

On va alors programmer un langage assembleur qui nous permettra de communiquer avec notre microprocesseur. Notamment, on programmera une horloge graphique (càd qu'on s'occupera
de l'affichage des pixels depuis notre microprocesseur)
pour tester notre microprocesseur.


\section{Les registres}

Tous les calculs du microprocceseur seront faits sur des registres de 16 bits.

Il y a en tout 8 registres.


\section{Représentation des instructions}

Le microprocesseur lit les instructions qu'il doit éxécuter depuis un fichier (lu avec l'instruction ROM)

Chaque instruction fait 16 bits, prend en argument 2 numéros de registre et une constante, et est découpée en

\begin{itemize}
    \item 4 bits du code de l'instruction
    \item 2*3 bits de numéro de registre
    \item 6 bits de constante
\end{itemize}


\section{Du langage assembleur à la ROM}

Un parseur menhir lit les instructions et écrit dans un fichier lisible par la ROM les différentes instructions

Voici une liste des différentes instructions, de leur code en binaire, et de leur effet.

\;



\begin{tabularx}{15cm}{|c|p{4cm}|X|}
    \hline
    nom & code & effet \\
    \hline
    stop & 0000 & stoppe le programme \\
    \hline
    add r1 r2 & 0001 & r1 <- r1 + r2 \\
    \hline
    sub r1 r2 & 0010 & r1 <- r1 - r2 \\
    \hline
    movr r1 r2 & 0011 & r1 <- r2 \\
    \hline
    movc r1 c & 0100 & r1 <- c \\
    \hline
    jump r1 r2 c & 0101 & on saute à la ligne c si r1 = r2 \\
    \hline
    getram r1 r2 c & 0110 & r1 <- RAM(r2 + c) \\
    \hline
    setram r1 r2 c & 0111 & RAM(r2 + c) <- r1 \\
    \hline
    rom r1 r2 c & 1000 & r1 <- ROM(r2 + c) \\
    \hline
\end{tabularx}







\end{document}