\documentclass[12pt,a4paper,french]{article}
\usepackage[utf8]{inputenc}
\usepackage[french]{babel}
\selectlanguage{french}
\usepackage[T1]{fontenc}
\usepackage{amssymb}
\usepackage{amsmath}

\title{Projet de simulateur de microprocesseur}
\author{Nathan Boyer, Matteo Torrents, Antoine Cuvelier, Grégoire Lecorre}


\begin{document}

\maketitle

\section*{Introduction}

On cherche à coder un simulateur de microprocesseur en Ocaml.

Pour cela, on va utiliser un simulateur de netlist qui éxécutera des netlists écrites en miniJazz.

Cette netlist peut éxécuter les opérations arithmétiques de base (and, or etc \ldots) sur un bit, manipuler des bus de bits (splice, concat \ldots)

De plus, l'instruction reg permet de se prendre la valeur du cycle précédent d'une variable. L'instruction RAM permet d'écrire et de lire dans la mémoire
(par soucis de réalisme, on considérera qu'il n'existe qu'une seule RAM). L'instruction ROM, quant à elle, permet de lire le contenu d'un fichier composé de 0 et de 1

On va alors programmer un langage assembleur qui nous permettra de communiquer avec notre microprocesseur. Notamment, on programmera une horloge textuelle, puis une horloge graphique
pour tester notre microprocesseur.





\end{document}